
\documentclass[10pt, oneside]{article}
\usepackage{amsmath, amsthm, amssymb, calrsfs, wasysym, verbatim, bbm, color, graphics, geometry}

\geometry{tmargin=.75in, bmargin=.75in, lmargin=.75in, rmargin = .75in}

\newcommand{\R}{\mathbb{R}}
\newcommand{\C}{\mathbb{C}}
\newcommand{\Z}{\mathbb{Z}}
\newcommand{\N}{\mathbb{N}}
\newcommand{\Q}{\mathbb{Q}}
\newcommand{\Cdot}{\boldsymbol{\cdot}}

\newtheorem{thm}{Theorem}
\newtheorem{defn}{Definition}
\newtheorem{conv}{Convention}
\newtheorem{rem}{Remark}
\newtheorem{lem}{Lemma}
\newtheorem{cor}{Corollary}


\title{ECON 6787: Introduction to CRD}
\author{Alejandro Ouslan}
\date{Academic Year 2025-2026}

\begin{document}

\maketitle
\tableofcontents

\vspace{.25in}

\section{Fall 2025}

\subsection{Introduction}

\begin{enumerate}
	\item In a completly randomized design, abbriviated as CRD, with one treatment
	      factor, n experimental units are divided randomly into $t$ treatments.
	\item if $n = tr$ is a multiple of $t$, then each level of the factor will be applied to $r$
	      unique experimental unites
	      \begin{enumerate}
		      \item there will be $r$ rteplicates of each run with the same level of the treatment factor.
		      \item if $n$ is not a multiple of $t$, then therre will be an unequal number of rteplicates
		            of each factor level.
	      \end{enumerate}
	\item All other known independent variables are help constant so that the will not bias as the effects.
	\item This design shuold be used when there is only one factor under study and the experimental unites are
	      homegenous.


\end{enumerate}


\end{document}
