\documentclass[10pt, oneside]{article}
\usepackage{amsmath, amsthm, amssymb, calrsfs, wasysym, verbatim, bbm, color, graphics, geometry}

\geometry{tmargin=.75in, bmargin=.75in, lmargin=.75in, rmargin = .75in}

\newcommand{\R}{\mathbb{R}}
\newcommand{\C}{\mathbb{C}}
\newcommand{\Z}{\mathbb{Z}}
\newcommand{\N}{\mathbb{N}}
\newcommand{\Q}{\mathbb{Q}}
\newcommand{\Cdot}{\boldsymbol{\cdot}}

\newtheorem{thm}{Theorem}
\newtheorem{defn}{Definition}
\newtheorem{conv}{Convention}
\newtheorem{rem}{Remark}
\newtheorem{lem}{Lemma}
\newtheorem{cor}{Corollary}


\title{Latin Square Designs: [ESMA 6787]}
\author{[Alejandro Ouslan]}
\date{Academic Year 2025-2026}

\begin{document}

\maketitle
\tableofcontents

\vspace{.25in}

\section{Introduciotn}
\begin{enumerate}
	\item Row-Collumn Design: units can reasonably be sorted by two
	      characteristics rather thatn one, and the most commonly used of
	      these are Latin Square Designs (LSD)
	\item The result is a two-way classification of units based on two
	      protential sources of nuisance variataion, so the unit-to-unit relationships
	      cannot be descirbed independently within rows ignoring columns, or
	      independetly withing columns ignoring rows.
	\item In order to ensure treatment-block balance comparable to that dfound in
	      CRBDs,
	      \begin{enumerate}
		      \item Design is a CRBD with respect to rows as blocks, ignoring columns
		      \item Designs is a CRD with respect to columns as block, ignoring rows.
	      \end{enumerate}
	\item This is , in fact, how a Lating Square Design is contructed.
\end{enumerate}

\section{Latin Squares design (LSD)}
\begin{enumerate}
	\item THe number of rwo-blocks of units must be  $t$ (the number of
	      treatments), since each treatment must appear exactly once in each column-block.
	\item The number of colus-blocks of units must be $t$, since each treatment
	      must appear exactly once in each row-block.
	\item THerefore, a LSD must contain a total of $t^2$ units, $t$ of which
	      must be assigned to each treatment, and result in $N=t^2$ data
	      values for each responce variable.
\end{enumerate}
$$
	\begin{bmatrix}
		1 & 2 & 3 \\
		2 & 3 & 1 \\
		3 & 1 & 2
	\end{bmatrix}
$$
$$
	\begin{bmatrix}
		1 & 2 & 3 & 4 \\
		2 & 3 & 1 & 2 \\
		3 & 1 & 2 & 2 \\
		8 & 1 & 2 & 1
	\end{bmatrix}
$$
\begin{enumerate}
	\item Since each experimetntal unit is consitain in two blocks,
	      randomizaiton is somewhat less straightfoward for LsDs thatn wiht
	      with CBDs.
	\item randomizaiton
\end{enumerate}

\section{Linear Model for LSD}
\subsection{ANOVA skeleton for LSD}
\subsection{Exampel: Operatiors and machine}
Consider a factory setting where you are producing a product with 4
operators and 4 machines. We call the columns the operators and
the rows the machines. Then you can randomly assign the specific
operators to a row and the specific macines to a column. The
treatment is one of four protocols for producing the product and our
interest is in the average time needed to produce each product.
Create a the ANOVA skeleton.
$$
	\begin{bmatrix}
		1 & 2 & 3 & 4 \\
		4 & 1 & 2 & 3 \\
		3 & 4 & 1 & 2 \\
		2 & 3 & 4 & 1
	\end{bmatrix}
$$

\subsection{Model in matix from}
Consider the follwing:
$$
	y = X_1\theta + X_2\tau + \epsilon
$$
\begin{enumerate}
	\item Modle matrix for $\mu, \alpha, \beta$
	      \[
		      \begin{split}
			      X_1 = []
		      \end{split}
	      \]
\end{enumerate}

$$
	\begin{array}{|c|c|c|}
		\hline
		1    & 2    & 3    \\
		\hline
		27.3 & 41.9 & 36.8 \\
		31.6 & 36.8 & 39.2 \\
		34.6 & 38.9 & 36.1 \\
		29.4 & 37.5 & 38.0 \\
		\hline
	\end{array}
$$

\subsection{R: example LSD (ARtificial data)}

\begin{enumerate}
	\item
\end{enumerate}.


\end{document}

