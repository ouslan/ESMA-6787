\documentclass[10pt, oneside]{article}
\usepackage{amsmath, amsthm, amssymb, calrsfs, wasysym, verbatim, bbm, color, graphics, geometry}

\geometry{tmargin=.75in, bmargin=.75in, lmargin=.75in, rmargin = .75in}

\newcommand{\R}{\mathbb{R}}
\newcommand{\C}{\mathbb{C}}
\newcommand{\Z}{\mathbb{Z}}
\newcommand{\N}{\mathbb{N}}
\newcommand{\Q}{\mathbb{Q}}
\newcommand{\Cdot}{\boldsymbol{\cdot}}

\newtheorem{thm}{Theorem}
\newtheorem{defn}{Definition}
\newtheorem{conv}{Convention}
\newtheorem{rem}{Remark}
\newtheorem{lem}{Lemma}
\newtheorem{cor}{Corollary}

\title{ESMA 6787: Homework 1}
\author{Alejandro M. Ouslan}
\date{Due Date: September 4, 2025}

\begin{document}

\maketitle

\vspace{.25in}

\section*{Problem 1: Syllabus Acknowledgment}
I have read the syllabus, understand its contents,and have no questions.

% Insert response here

\section*{Problem 2: Definitions in Your Own Words}
\begin{itemize}
	\item \textbf{Experiment:} Method to answer a research question, where the researcher controles an
	      environment and only changes one or many treatment holding all other things constant
	\item \textbf{Experimental unit:} unit that will receive the treatment
	\item \textbf{Observational unit:}  it is the person, thing or event where are studying.
	\item \textbf{Background variable:} Are all the factor that can affect the study but are not controlled for
	\item \textbf{Independent (predictor) variable:} It is what is applied as treatment
	\item \textbf{Dependent (response) variable:} It is the outcome or what we observe after we apply the treatment
	\item \textbf{Confounded factors:} Are un observe factors that affect both the treatment and the observed factor
	\item \textbf{Experimental error:} It is how far our experiment is from the reality.
	\item \textbf{Randomization:} Treatments are applied randomly
	\item \textbf{Replicate:} recreation of the experiments with the same settings and specification but with different data
\end{itemize}

\section*{Problem 3: Lady Tasting Tea}
\begin{itemize}
	\item[(a)] Units in this experiment: The tea cups
	\item[(b)] Treatments in this experiment: The order of what was added first to the tea
	\item[(c)] Randomization method using physical devices: order cups from 1-8 then randomize the order with a program
	\item[(d)] Adjustments if cups differ in material (porcelain vs china): Assigned cups type randomly to each group
\end{itemize}

\section*{Problem 4: Paper Airplane Experiment}
\begin{itemize}
	\item[(a)] Experimental treatments: The experiments would be each of the 3 recipe, with the simple classic airplane as the controle group
	\item[(b)] Experimental units and homogeneity: to ensure homogeneity we would use the same type paper, thrown from the same location.
	\item[(c)] Randomization process: when we get the 20 papers we assign the folding recepte randomly to them
	\item[(d)] Procedure for applying treatment to unit: when the recepte is chacen use the as much possible machines or tools ensure cons
	\item[(e)] Measurement process: Mesure the distance the plane flew
\end{itemize}

\section*{Problem 5: Gasoline Mileage Study}
\begin{itemize}
	\item[(a)] Comparison of strengths and weaknesses:
	      Design experiment A is convenient but their are many variables that are not controlled like the
	      brand of car, driving habits, etc. Design B has more strengths given that we use the same care for the same amount of time but a problem that could
	      have the experiment is segregation of the vie vale over time. Design C addresses the concerns of the previouse experiemnt.
	\item[(b)] Identification of true experiment(s) and justification:
	      Experiment C would say it is the best experiment given that it controles for the most amount of factors buy using the same vehicle.
\end{itemize}

\section*{Problem 6: Baseball League as Experiment}
\begin{itemize}
	\item Treatments and units: The teams and the unites are the games
	\item Application of treatment to unit: The application of the treatment would be to
	      that a particular team is playing at its expectations of victory is the outcome
	\item Randomization and replication: Randomizing the matching of the team and ensuring that each team plays each other for the replications
	\item Possibility and use of blocking: Could be used if there are confounding factors like home-field advantage or team skill differences, but it might not be necessary if teams are assumed to be comparable.
\end{itemize}

\section*{Problem 7: Tomato Fertilizer and Variety}
\begin{itemize}
	\item Experimental setup: We would devide the plot of land into quadrents and each will have a groups with all combinations of Fertilizer and seed type.
	      In addition it will consist of a control group with no Fertilizer and a comon tomato seed.
	\item Use of replication and randomization: The randomization is applied to which section of the plot of land receives the combination of Fertilizer and seed
	\item Additional design principles in second season: rotate the treatments, long-term replication, seasonal variation consideration, increased precision in measurement and control of weather variability.
\end{itemize}

\section*{Problem 8: Hand Washing Experiment}
\begin{itemize}
	\item[(a)] Experimental unit: The experimental unit is the individual subject (person) participating in the hand washing experiment
	\item[(b)] Factors: Wash water temperature, detergent concentration
	\item[(c)] Response:The response is the bacterial count on the palms of the subjects.
\end{itemize}

\section*{Problem 9: Real-life Application}
I would say most economic problem would benefit form experiments designs given that it changes the results from corrolation to causality.

\section*{Problem 10: Variance as Quadratic Form}
Suppose $y_1,\ldots,y_n \sim N(\mu,\sigma^2)$. Let $y = (y_1,\ldots,y_n)'$ and let $\bar{y} = \frac{1}{n}\sum^n_{i=1}y_i$. Show that
$\frac{1}{n-1}\sum^n_{i=1}(y_i - \bar{y})^2$ can be written as $y'Ay$ for sum matrix $A$. Identify the A matrix.
\[
	\frac{1}{n-1}(y - \bar{y}\mathbf{1})'(y - \bar{y}\mathbf{1})
\]

\[
	y - \bar{y}\mathbf{1} = y - \left( \frac{1}{n} \mathbf{1}'y \right)\mathbf{1} = \left( I - \frac{1}{n} \mathbf{1}\mathbf{1}' \right)y
\]

\[
	\frac{1}{n-1} (y - \bar{y}\mathbf{1})'(y - \bar{y}\mathbf{1}) = \frac{1}{n-1} y' C'C y
\]

\[
	\frac{1}{n-1} y' C y
\]
\[
	A = \frac{1}{n-1} \left( I - \frac{1}{n} \mathbf{1}\mathbf{1}' \right)
\]

\section*{Problem 11: Cell Means Model with Unequal Group Sizes}
consider a cells means model with $T=4$ treatments and $n_1=n_2=5,n_3=3,n_4=7$
\begin{itemize}
	\item[(a)] Proposed cell means model:
	      \[
		      y_{ij} = \mu_i + \varepsilon_{ij}, \quad i = 1, 2, 3, 4
	      \]
	\item[(b)] Design matrix $X$ and its rank:
	      \[
		      X = \begin{bmatrix}
			      1 & 0 & 0 & 0 \\
			      1 & 0 & 0 & 0 \\
			      1 & 0 & 0 & 0 \\
			      1 & 0 & 0 & 0 \\
			      1 & 0 & 0 & 0 \\
			      0 & 1 & 0 & 0 \\
			      0 & 1 & 0 & 0 \\
			      0 & 1 & 0 & 0 \\
			      0 & 0 & 1 & 0 \\
			      0 & 0 & 1 & 0 \\
			      0 & 0 & 0 & 1 \\
			      0 & 0 & 0 & 1 \\
			      0 & 0 & 0 & 1 \\
			      0 & 0 & 0 & 1 \\
			      0 & 0 & 0 & 1 \\
			      0 & 0 & 0 & 1 \\
			      0 & 0 & 0 & 1 \\
		      \end{bmatrix}, \quad \text{rank}(X) = 4
	      \]
	\item[(c)] Computation of $X'X$:
	      \[
		      X'X = \begin{bmatrix}
			      5 & 0 & 0 & 0 \\
			      0 & 5 & 0 & 0 \\
			      0 & 0 & 3 & 0 \\
			      0 & 0 & 0 & 7 \\
		      \end{bmatrix}
	      \]
	\item[(d)] OLS estimates as a function of $y$, i.e, $\hat{\beta}=(X'X)_-X'y$:
	      \[
		      \hat{\beta} = \begin{bmatrix}
			      \frac{1}{5} \sum_{j=1}^{5} y_{1j} \\
			      \frac{1}{5} \sum_{j=1}^{5} y_{2j} \\
			      \frac{1}{3} \sum_{j=1}^{3} y_{3j} \\
			      \frac{1}{7} \sum_{j=1}^{7} y_{4j} \\
		      \end{bmatrix}
	      \]
	\item[(e)] Analysis of Table 1 dataset:
	      \begin{itemize}
		      \item[i.] OLS estimates:
		            \[
			            \hat{\mu}_1 = \frac{4.25 + 3.92 + 8.48 + 6.66 + 1.15}{5} = 4.89,
		            \]
		            \[
			            \hat{\mu}_2 = \frac{9.55 + 6.08 + 11.53 + 4.78 + 6.18}{5} = 7.224
		            \]
		            \[
			            \hat{\mu}_3 = \frac{15.74 + 11.70 + 10.76}{3} = 12.7333,
		            \]
		            \[
			            \hat{\mu}_4 = \frac{14.85 + 14.34 + 13.89 + 14.40 + 15.53 + 12.74 + 12.97}{7} = 14.103
		            \]
		      \item[ii.] Projection matrix $P_X$:
		            \[
			            P_X = X(X'X)^{-1}X', \quad \text{where } X \text{ is the design matrix from part (b)}
		            \]
		            Since \(X'X\) is diagonal, its inverse is:
		            \[
			            (X'X)^{-1} = \begin{bmatrix}
				            \frac{1}{5} & 0           & 0           & 0           \\
				            0           & \frac{1}{5} & 0           & 0           \\
				            0           & 0           & \frac{1}{3} & 0           \\
				            0           & 0           & 0           & \frac{1}{7}
			            \end{bmatrix}
		            \]
		            So, \(P_X\) projects each observation to its group mean.

		      \item[iii.] Compute $y'(I - P_X)y$:
		            \[
			            y'(I - P_X)y = \sum_{i=1}^{4} \sum_{j=1}^{n_i} (y_{ij} - \hat{\mu}_i)^2
		            \]
		            \[
			            = \sum_{j=1}^{5} (y_{1j} - 4.89)^2 + \sum_{j=1}^{5} (y_{2j} - 7.224)^2 + \sum_{j=1}^{3} (y_{3j} - 12.7333)^2 + \sum_{j=1}^{7} (y_{4j} - 14.103)^2 = 111.0322
		            \]
		      \item[iv.] Compute $\bar{y}_{..}$ and $\bar{y}_i.$ for all $i$:
		            \[
			            \bar{y}_{..} = \frac{1}{20} \sum y = \frac{261.77}{20} = 13.0885
		            \]
		            \[
			            \bar{y}_{1.} = 4.89, \quad \bar{y}_{2.} = 7.224, \quad \bar{y}_{3.} = 12.7333, \quad \bar{y}_{4.} = 14.103
		            \]
		      \item[v.] Estimability of $\mu_1$:
		            \[
			            \mu_1 \text{ is estimable since it corresponds to a column in } X
		            \]
		      \item[vi.] Estimability of $\mu_2 - \mu_3$:
		            \[
			            \mu_2 - \mu_3 \text{ is estimable since both } \mu_2 \text{ and } \mu_3 \text{ are estimable}
		            \]
		      \item[vii.] Estimability of $\mu_1 - \frac{\mu_2 + \mu_3}{2}$:
		            \[
			            \mu_1 - \frac{\mu_2 + \mu_3}{2} \text{ is a linear combination of estimable functions} \Rightarrow \text{estimable}
		            \]
	      \end{itemize}
\end{itemize}

\section*{Problem 12: Fixed-Effect Model with Unequal Group Sizes}
consider a cells means model with $T=4$ treatments and $n_1=n_2=5,n_3=3,n_4=7$
\begin{itemize}
	\item[(a)] Proposed fixed-effect model:
	      \[
		      y_{ij} = \mu + \alpha_i + \varepsilon_{ij}, \quad i = 1, 2, 3, 4,\quad j = 1, \dots, n_i
	      \]
	\item[(b)] Design matrix $X$ and its rank:
	      \[
		      X = \begin{bmatrix}
			      1 & 1 & 0 & 0 \\
			      1 & 1 & 0 & 0 \\
			      1 & 1 & 0 & 0 \\
			      1 & 1 & 0 & 0 \\
			      1 & 1 & 0 & 0 \\
			      1 & 0 & 1 & 0 \\
			      1 & 0 & 1 & 0 \\
			      1 & 0 & 1 & 0 \\
			      1 & 0 & 1 & 0 \\
			      1 & 0 & 1 & 0 \\
			      1 & 0 & 0 & 1 \\
			      1 & 0 & 0 & 1 \\
			      1 & 0 & 0 & 1 \\
			      1 & 0 & 0 & 1 \\
			      1 & 0 & 0 & 1 \\
			      1 & 0 & 0 & 1 \\
			      1 & 0 & 0 & 1 \\
		      \end{bmatrix}, \quad \text{rank}(X) = 4
	      \]
	\item[(c)] Computation of $X'X$:
	      \[
		      X'X =
		      \begin{bmatrix}
			      20 & 5 & 5 & 7 \\
			      5  & 5 & 0 & 0 \\
			      5  & 0 & 5 & 0 \\
			      7  & 0 & 0 & 7 \\
		      \end{bmatrix}
	      \]
	\item[(d)] OLS estimates as a function of $y$:
	      \[
		      \hat{\beta} = (X'X)^{-1} X'y
	      \]
	\item[(e)] Using Table 1 dataset:
	      \begin{itemize}
		      \item[i.] OLS estimates:
		            \[
			            \hat{\mu} = \bar{y}_{..} = 13.0885
		            \]
		            \[
			            \hat{\alpha}_i = \bar{y}_{i.} - \bar{y}_{..}
		            \]
		            \[
			            \begin{aligned}
				            \hat{\alpha}_1 & = 4.89 - 13.0885 = -8.1985    \\
				            \hat{\alpha}_2 & = 7.224 - 13.0885 = -5.8645   \\
				            \hat{\alpha}_3 & = 12.7333 - 13.0885 = -0.3552 \\
				            \hat{\alpha}_4 & = 14.103 - 13.0885 = 1.0145   \\
			            \end{aligned}
		            \]
		      \item[ii.] Projection matrix $P_X$:
		            \[
			            P_X = X (X'X)^{-1} X'
		            \]
		      \item[iii.] Compute $y'(I - P_X)y$:
		            \[
			            y'(I - P_X)y = \sum_{i=1}^4 \sum_{j=1}^{n_i} (y_{ij} - \hat{\mu} - \hat{\alpha}_i)^2 = 111.0322
		            \]
		      \item[iv.] Compute $\bar{y}_{..}$ and $\bar{y}_i.$ for all $i$:
		            \[
			            \bar{y}_{..} = \frac{1}{20} \sum y = 13.0885
		            \]
		            \[
			            \bar{y}_{1.} = 4.89, \quad \bar{y}_{2.} = 7.224, \quad \bar{y}_{3.} = 12.7333, \quad \bar{y}_{4.} = 14.103
		            \]
		      \item[v.] Estimability of $\alpha_1$:
		            no?
		      \item[vi.] Estimability of $\alpha_2 - \alpha_3$:
		            Yes, \(\alpha_2 - \alpha_3\)
		      \item[vii.] Estimability of $\alpha_1 - \frac{\alpha_3 + \alpha_4}{2}$:
		            Yes, this is a linear combination of estimable contrasts,
	      \end{itemize}
\end{itemize}

\end{document}
