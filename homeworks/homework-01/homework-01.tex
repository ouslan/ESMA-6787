\documentclass[10pt, oneside]{article}
\usepackage{amsmath, amsthm, amssymb, calrsfs, wasysym, verbatim, bbm, color, graphics, geometry}

\geometry{tmargin=.75in, bmargin=.75in, lmargin=.75in, rmargin = .75in}

\newcommand{\R}{\mathbb{R}}
\newcommand{\C}{\mathbb{C}}
\newcommand{\Z}{\mathbb{Z}}
\newcommand{\N}{\mathbb{N}}
\newcommand{\Q}{\mathbb{Q}}
\newcommand{\Cdot}{\boldsymbol{\cdot}}

\newtheorem{thm}{Theorem}
\newtheorem{defn}{Definition}
\newtheorem{conv}{Convention}
\newtheorem{rem}{Remark}
\newtheorem{lem}{Lemma}
\newtheorem{cor}{Corollary}

\title{ESMA 6787: Homework 1}
\author{Alejandro M. Ouslan}
\date{Due Date: September 4, 2025}

\begin{document}

\maketitle

\vspace{.25in}

\section*{Problem 1: Syllabus Acknowledgment}
I have read the syllabus, understand its contents,and have no questions.

% Insert response here

\section*{Problem 2: Definitions in Your Own Words}
\begin{itemize}
	\item \textbf{Experiment:} Method to answer a research question, where the researcher controles an 
    environment and only changes one or many treatment holding all other things constant
	\item \textbf{Experimental unit:} unit that will receive the treatment
	\item \textbf{Observational unit:}  it is the person, thing or event where are studying. 
	\item \textbf{Background variable:} Are all the factor that can affect the study but are not controlled for 
	\item \textbf{Independent (predictor) variable:} It is what is applied as treatment
	\item \textbf{Dependent (response) variable:} It is the outcome or what we observe after we apply the treatment
	\item \textbf{Confounded factors:} Are un observe factors that affect both the treatment and the observed factor
	\item \textbf{Experimental error:} It is how far our experiment is from the reality.
	\item \textbf{Randomization:} Treatments are applied randomly
	\item \textbf{Replicate:} recreation of the experiments with the same settings and specification but with different data 
\end{itemize}

\section*{Problem 3: Lady Tasting Tea}
\begin{itemize}
	\item[(a)] Units in this experiment:
	\item[(b)] Treatments in this experiment:
	\item[(c)] Randomization method using physical devices:
	\item[(d)] Adjustments if cups differ in material (porcelain vs china):
\end{itemize}

\section*{Problem 4: Paper Airplane Experiment}
\begin{itemize}
	\item[(a)] Experimental treatments:
	\item[(b)] Experimental units and homogeneity:
	\item[(c)] Randomization process:
	\item[(d)] Procedure for applying treatment to unit:
	\item[(e)] Measurement process:
\end{itemize}

\section*{Problem 5: Gasoline Mileage Study}
\begin{itemize}
	\item[(a)] Comparison of strengths and weaknesses:
	\item[(b)] Identification of true experiment(s) and justification:
\end{itemize}

\section*{Problem 6: Baseball League as Experiment}
\begin{itemize}
	\item Treatments and units:
	\item Application of treatment to unit:
	\item Randomization and replication:
	\item Possibility and use of blocking:
\end{itemize}

\section*{Problem 7: Tomato Fertilizer and Variety}
\begin{itemize}
	\item Experimental setup:
	\item Use of replication and randomization:
	\item Additional design principles in second season:
\end{itemize}

\section*{Problem 8: Hand Washing Experiment}
\begin{itemize}
	\item[(a)] Experimental unit:
	\item[(b)] Factors:
	\item[(c)] Response:
\end{itemize}

\section*{Problem 9: Real-life Application}
% Describe your situation and how an experimental design could improve prediction

\section*{Problem 10: Variance as Quadratic Form}
% Show the derivation and identify the matrix A

\section*{Problem 11: Cell Means Model with Unequal Group Sizes}
\begin{itemize}
	\item[(a)] Proposed cell means model:
	\item[(b)] Design matrix $X$ and its rank:
	\item[(c)] Computation of $X'X$:
	\item[(d)] OLS estimates as a function of $y$:
	\item[(e)] Analysis of Table 1 dataset:
	\begin{itemize}
		\item[i.] OLS estimates:
		\item[ii.] Projection matrix $P_X$:
		\item[iii.] Compute $y'(I - P_X)y$:
		\item[iv.] Compute $\bar{y}_{..}$ and $\bar{y}_i.$ for all $i$:
		\item[v.] Estimability of $\mu_1$:
		\item[vi.] Estimability of $\mu_2 - \mu_3$:
		\item[vii.] Estimability of $\mu_1 - \frac{\mu_2 + \mu_3}{2}$:
	\end{itemize}
\end{itemize}

\section*{Problem 12: Fixed-Effect Model with Unequal Group Sizes}
\begin{itemize}
	\item[(a)] Proposed fixed-effect model:
	\item[(b)] Design matrix $X$ and its rank:
	\item[(c)] Computation of $X'X$:
	\item[(d)] OLS estimates as a function of $y$:
	\item[(e)] Using Table 1 dataset:
	\begin{itemize}
		\item[i.] OLS estimates:
		\item[ii.] Projection matrix $P_X$:
		\item[iii.] Compute $y'(I - P_X)y$:
		\item[iv.] Compute $\bar{y}_{..}$ and $\bar{y}_i.$ for all $i$:
		\item[v.] Estimability of $\alpha_1$:
		\item[vi.] Estimability of $\alpha_2 - \alpha_3$:
		\item[vii.] Estimability of $\alpha_1 - \frac{\alpha_3 + \alpha_4}{2}$:
	\end{itemize}
\end{itemize}

\end{document}
