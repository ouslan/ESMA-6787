\documentclass{article}

\usepackage{fancyhdr}
\usepackage{extramarks}
\usepackage{amsmath}
\usepackage{amsthm}
\usepackage{amsfonts}
\usepackage{tikz}
\usepackage[plain]{algorithm}
\usepackage{algpseudocode}

\usetikzlibrary{automata,positioning}

%
% Basic Document Settings
%

\topmargin=-0.45in
\evensidemargin=0in
\oddsidemargin=0in
\textwidth=6.5in
\textheight=9.0in
\headsep=0.25in

\linespread{1.1}

\pagestyle{fancy}
\lhead{\hmwkAuthorName}
\chead{\hmwkClass\ (\hmwkClassInstructor): \hmwkTitle}
\rhead{\firstxmark}
\lfoot{\lastxmark}
\cfoot{\thepage}

\renewcommand\headrulewidth{0.4pt}
\renewcommand\footrulewidth{0.4pt}

\setlength\parindent{0pt}

%
% Create Problem Sections
%

\newcommand{\enterProblemHeader}[1]{
	\nobreak\extramarks{}{Problem \arabic{#1} continued on next page\ldots}\nobreak{}
	\nobreak\extramarks{Problem \arabic{#1} (continued)}{Problem \arabic{#1} continued on next page\ldots}\nobreak{}
}

\newcommand{\exitProblemHeader}[1]{
	\nobreak\extramarks{Problem \arabic{#1} (continued)}{Problem \arabic{#1} continued on next page\ldots}\nobreak{}
	\stepcounter{#1}
	\nobreak\extramarks{Problem \arabic{#1}}{}\nobreak{}
}

\setcounter{secnumdepth}{0}
\newcounter{partCounter}
\newcounter{homeworkProblemCounter}
\setcounter{homeworkProblemCounter}{1}
\nobreak\extramarks{Problem \arabic{homeworkProblemCounter}}{}\nobreak{}

%
% Homework Problem Environment
%
% This environment takes an optional argument. When given, it will adjust the
% problem counter. This is useful for when the problems given for your
% assignment aren't sequential. See the last 3 problems of this template for an
% example.
%
\newenvironment{homeworkProblem}[1][-1]{
	\ifnum#1>0
		\setcounter{homeworkProblemCounter}{#1}
	\fi
	\section{Problem \arabic{homeworkProblemCounter}}
	\setcounter{partCounter}{1}
	\enterProblemHeader{homeworkProblemCounter}
}{
	\exitProblemHeader{homeworkProblemCounter}
}

%
% Homework Details
%   - Title
%   - Due date
%   - Class
%   - Section/Time
%   - Instructor
%   - Author
%

\newcommand{\hmwkTitle}{Asignacion\ \#2}
\newcommand{\hmwkDueDate}{October 10, 2025}
\newcommand{\hmwkClass}{ESMA 6787}
\newcommand{\hmwkClassInstructor}{Israel Almodovar}
\newcommand{\hmwkAuthorName}{\textbf{Alejandro Ouslan}}

%
% Title Page
%

\title{
	\vspace{2in}
	\textmd{\textbf{\hmwkClass:\ \hmwkTitle}}\\
	\normalsize\vspace{0.1in}\small{Due\ on\ \hmwkDueDate}\\
	\vspace{0.1in}\large{\textit{\hmwkClassInstructor}}
	\vspace{3in}
}

\author{\hmwkAuthorName}
\date{}

\renewcommand{\part}[1]{\textbf{\large Part \Alph{partCounter}}\stepcounter{partCounter}\\}

%
% Various Helper Commands
%

% Useful for algorithms
\newcommand{\alg}[1]{\textsc{\bfseries \footnotesize #1}}

% For derivatives
\newcommand{\deriv}[1]{\frac{\mathrm{d}}{\mathrm{d}x} (#1)}

% For partial derivatives
\newcommand{\pderiv}[2]{\frac{\partial}{\partial #1} (#2)}

% Integral dx
\newcommand{\dx}{\mathrm{d}x}

% Alias for the Solution section header
\newcommand{\solution}{\textbf{\large Solution}}

% Probability commands: Expectation, Variance, Covariance, Bias
\newcommand{\E}{\mathrm{E}}
\newcommand{\Var}{\mathrm{Var}}
\newcommand{\Cov}{\mathrm{Cov}}
\newcommand{\Bias}{\mathrm{Bias}}

\begin{document}

\maketitle

\pagebreak

% Homework problem 1
\begin{homeworkProblem}
	Define using your own words:
	\begin{enumerate}
		\item Estimate $c'\beta$
		\item Power
		\item Null hypothesis
		\item Alternative hypothesis
		\item Test Statistics
		\item Non-centrality parameter
		\item Details about the non-centrality parameter
	\end{enumerate}
\end{homeworkProblem}

% Homework problem 2
\begin{homeworkProblem}
	Consider a completely randomized design with four treatment groups, with $n_i>0$ units assigned to treatment $i=1,2,3,4$.
	\begin{enumerate}
		\item One way to model data from such an experiment is with the effect model:
		      $$
			      y_{ij}=\alpha + \tau_i + \epsilon{ij}, \quad i=1,2,3,4; \quad j= 1.\ldots, n_i
		      $$
		      Under this model show why each of the following is estimable or nonestimable:
		      $$
			      \tau_3, \tau_3 - \tau_2, \tau_3 + \tau_2
		      $$
		\item Now define a different model for the same experiment, as:
		      $$
			      y_{1j}= \mu_1 + \epsilon_{1j}, \quad j=1,\ldots,n_i
		      $$
		      $$
			      y_{ij}= \mu_1 + \theta_i + \epsilon_{ij} \quad i=2,3,4; \quad j=1,\ldots,n_i
		      $$
		      Under this model, show why each of the followings is estimable or nonestimable:
		      $$
			      \theta_3,\theta_3-\theta_2, \theta_3 + \theta_2
		      $$
	\end{enumerate}
\end{homeworkProblem}

% Homework problem 3
\begin{homeworkProblem}

	A chemical engineer is interested in comparing three different versions of a reaction process, labeled
	A, B, and C, with respect to “percent conversion of feedstock.” In a preliminary experiment, she
	applied each process to four batches of raw material, using appropriate randomization of the 12
	available batches to the three treatments, and collected the percent conversion values presented in
	the following table.

	$$
		\begin{array}{|c|c|c|}
			\hline
			A    & B    & C    \\
			\hline
			27.3 & 41.9 & 36.8 \\
			31.6 & 36.8 & 39.2 \\
			34.6 & 38.9 & 36.1 \\
			29.4 & 37.5 & 38.0 \\
			\hline
		\end{array}
	$$

	Assuming the data are independent and can be reasonably modeled as:
	$$
		y_{ij}= \mu_i + \epsilon{ij}, \quad E(e_{ij})=0, \quad Var(e_{ij})= \sigma^2
	$$
	\begin{enumerate}
		\item Estimate $\sigma^2$ and test the hypothesis: $\mu_i = \mu_2 = \mu_3$ against $H_1$: at least one of $\mu_i$ is different.
		\item Using your estimate of $\sigma$ as if it were the true parameter value, how large would a follow-up experiment
		      (with equal sample sizes) have to be so that the $0.05$-level confidence interval for $\mu_1 - \mu_3$ would have expected width ($5\%$).
	\end{enumerate}

\end{homeworkProblem}

% Homework problem 4
\begin{homeworkProblem}
	Consider a completely randomized design with five treatment groups, in which a total of $N = 50$
	units are to be used. Although it won’t be explicitly used in the analysis model, treatments $1$
	through $5$ actually represent increasing concentrations of one component in an otherwise standard
	chemical compound, and the primary purpose of the experiment is to understand whether certain
	measurable properties of the compound change with this concentration. The investigator decides
	to address these questions by estimating four quantities:
	$$
		\tau_2 - \tau_1,\tau_3-\tau_2,\tau_4-\tau_3,\tau_5-\tau_4
	$$
	where each $\tau_i$ is a parameter in the standard effects model. Find the optimal allocation for the $50$
	available units (i.e., values for $n_1,\ldots , n_5$) that minimizes the average variance of estimates of the
	four contrasts of interest. Do this as a constrained, continuous optimization problem, then round
	the solution to integer values that are consistent with the required constraint.
\end{homeworkProblem}

% Homework problem 5
\begin{homeworkProblem}
	Continue working with the experimental design described in problem 2. Suppose the experiment-
	specific treatment means in this problem, as would be expressed in the cell means model, are
	actually:
	$$
		\begin{array}{|c|c|c|c|c|}
			\hline
			\mu_1 & \mu_2 & \mu_3 & \mu_4 & \mu_5 \\
			\hline
			10    & 11    & 12    & 12    & 12    \\
			\hline
		\end{array}
	$$
	and $\sigma=2$. What is the power of the standard F-test for the hypothesis
	$$
		\tau_1 = \tau_2 = \tau_3 = \tau_4 = \tau_5
	$$
	at $\alpha=0.05$:
	\begin{enumerate}
		\item if all $n_i=10$?
		\item under the optimal sample sample allocation you found in problem 2?
		\item Derive an optimal allocation for the F-test of equal treatment effects, i.e, the sample size
		      (totaling 50) that would result in the greatest power, if in reality the experiment-specific means are
		      $\mu_1=10$ and $\tau_2 = \tau_3 = \tau_4 = \tau_5 =8$
	\end{enumerate}
\end{homeworkProblem}

% Homework problem 6
\begin{homeworkProblem}
	The entire class of ESMA 6616 wanted to study how the power of the F changes with the non-
	centrality parameter $\lambda^2$. To achieve this we will do as follow. Consider the model
	$$
		y_{ij} = \mu_i + \epsilon_{ij}
	$$
	Where $\epsilon_{ij}\sim N(0,\sigma^2)$, with $i=1,\ldots,k$ and $j=1,...,n_i$.

	The null and alternative hypotheses for the ANOVA F-test are
	\[
		\begin{split}
			H_0 & : \mu_1 = \ldots = \mu_k           \\
			H_1 & : \text{at least one is different}
		\end{split}
	\]
	The test rejects $H_0$ if the $F^\star$-statistic, defined as $F^\star = \frac{MS_{model}}{MS_{error}}$ , exceeds
	a critical value.

	The power of the ANOVA F-test, which measures the probability of rejecting $H_0$ when $H_1$ is true, is
	given by $P(F^\star > \alpha|H_1)$. To calculate the power, we must know the distribution
	of $F\star$. Under $H_0$, $F^\star \sim F_{k-1,\sum_{k}^{i=1}n_i-k}$
	degrees of freedom. The distribution under $H_1$ depends on
	the true differences among the group means.

	Consider the following example $k = 5$ treatment groups with group sizes $n_1 = n_2 = n_3 = 10, n_4 = 8$,
	and $n_5 = 6$. Further, assume the standard deviation $\sigma = 1.5$. The task is to find the power of the
	test when $H_1$ is true, given the group means $\mu_1 = 2, \mu_2 = 3, \mu_3 = 2.5, \mu_4 = 0$, and $\mu_5 = 1$. Assume
	we are using a significance level of 0.05.

	The overall mean is defined $\bar{\mu}= \sum_{5}^{i=1}n_i*\frac{\mu_i}{\sum_{5}^{i=1}*n_i}$

	The non-centrality parameter is $\frac{\sum_{5}^{i=1}n_i(\mu_i-\bar{\mu})^2}{\sigma^2}$

	In R, the argument ncp stands for the non-centrality parameter in the desity functions. In this example, the test statisic
	have the following distribution $F^\star \sim F_{5-1,40-5}(\lambda^2=0)$ under $H_0$ and $F^\star \sim F_{5-1,40-5}(\lambda^2 = 15.32222)$ under $H_1$.

	Figure 1 showed the difference between the F distribution under the null hypothesis(solid red) with
	4 and 35 with non-centrality parameter of 0 and the F under the alternative (dashed blue) with
	the same degrees of freedom but non-centrality parameter $\lambda^2 = 15.3222$. We can observed as $\lambda^2$
	increases the density under the alternative get farther from the null. The critical value to reject $H_0$
	keeping the significance level $2.6415$. If $H_1$ is true and $F^\star \sim F_{5-1,40-5}(\lambda^2 = 15.32222)$, the power
	to reject $H_0$ is the

\end{homeworkProblem}

% Homework problem 7
\begin{homeworkProblem}

\end{homeworkProblem}

\end{document}
