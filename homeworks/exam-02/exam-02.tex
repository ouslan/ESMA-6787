\documentclass{article}

\usepackage{fancyhdr}
\usepackage{extramarks}
\usepackage{amsmath}
\usepackage{amsthm}
\usepackage{amsfonts}
\usepackage{tikz}
\usepackage[plain]{algorithm}
\usepackage{algpseudocode}
\usepackage{listings}
\usepackage{booktabs}
\usepackage{xcolor}
\usepackage[english]{babel}
\usepackage[T1]{fontenc}
\usepackage{lmodern,mathrsfs}
\usepackage{xparse}
\usepackage[inline,shortlabels]{enumitem}
\setlist{topsep=2pt,itemsep=2pt,parsep=0pt,partopsep=0pt}
\usepackage[dvipsnames]{xcolor}
\usepackage[utf8]{inputenc}
\usepackage[a4paper,top=0.5in,bottom=0.2in,left=0.5in,right=0.5in,footskip=0.3in,includefoot]{geometry}
\usepackage[most]{tcolorbox}
\tcbuselibrary{minted} % tcolorbox minted library, required to use the "minted" tcb listing engine (this library is not loaded by the option [most])
\usepackage{minted} % Allows input of raw code, such as Python code
% \usepackage[colorlinks]{hyperref}


\usetikzlibrary{automata,positioning}

\tcbset{
    pythoncodebox/.style={
        enhanced jigsaw,breakable,
        colback=gray!10,colframe=gray!20!black,
        boxrule=1pt,top=2pt,bottom=2pt,left=2pt,right=2pt,
        sharp corners,before skip=10pt,after skip=10pt,
        attach boxed title to top left,
        boxed title style={empty,
            top=0pt,bottom=0pt,left=2pt,right=2pt,
            interior code={\fill[fill=tcbcolframe] (frame.south west)
                --([yshift=-4pt]frame.north west)
                to[out=90,in=180] ([xshift=4pt]frame.north west)
                --([xshift=-8pt]frame.north east)
                to[out=0,in=180] ([xshift=16pt]frame.south east)
                --cycle;
            }
        },
        title={#1}, % Argument of pythoncodebox specifies the title
        fonttitle=\sffamily\bfseries
    },
    pythoncodebox/.default={}, % Default is No title
    %%% Starred version has no frame %%%
    pythoncodebox*/.style={
        enhanced jigsaw,breakable,
        colback=gray!10,coltitle=gray!20!black,colbacktitle=tcbcolback,
        frame hidden,
        top=2pt,bottom=2pt,left=2pt,right=2pt,
        sharp corners,before skip=10pt,after skip=10pt,
        attach boxed title to top text left={yshift=-1mm},
        boxed title style={empty,
            top=0pt,bottom=0pt,left=2pt,right=2pt,
            interior code={\fill[fill=tcbcolback] (interior.south west)
                --([yshift=-4pt]interior.north west)
                to[out=90,in=180] ([xshift=4pt]interior.north west)
                --([xshift=-8pt]interior.north east)
                to[out=0,in=180] ([xshift=16pt]interior.south east)
                --cycle;
            }
        },
        title={#1}, % Argument of pythoncodebox specifies the title
        fonttitle=\sffamily\bfseries
    },
    pythoncodebox*/.default={}, % Default is No title
}

% Custom tcolorbox for Python code (not the code itself, just the box it appears in)
\newtcolorbox{pythonbox}[1][]{pythoncodebox=#1}
\newtcolorbox{pythonbox*}[1][]{pythoncodebox*=#1} % Starred version has no frame

% Basic Document Settings
\topmargin=-0.45in
\evensidemargin=0in
\oddsidemargin=0in
\textwidth=6.5in
\textheight=9.0in
\headsep=0.25in
\linespread{1.1}

\pagestyle{fancy}
\lhead{\hmwkAuthorName}
\chead{\hmwkClass\ (\hmwkClassInstructor): \hmwkTitle}
\rhead{\firstxmark}
\lfoot{\lastxmark}
\cfoot{\thepage}
\renewcommand\headrulewidth{0.4pt}
\renewcommand\footrulewidth{0.4pt}
\setlength\parindent{0pt}

% Homework Details
\newcommand{\hmwkTitle}{Exam 2}
\newcommand{\hmwkDueDate}{December 14, 2025}
\newcommand{\hmwkClass}{ESMA 6787}
\newcommand{\hmwkClassInstructor}{Israel Almodovar}
\newcommand{\hmwkAuthorName}{\textbf{Alejandro Ouslan}}

% Title Page
\title{
	\vspace{2in}
	\textmd{\textbf{\hmwkClass:\ \hmwkTitle}}\\
	\normalsize\vspace{0.1in}\small{Due\ on\ \hmwkDueDate}\\
	\vspace{0.1in}\large{\textit{\hmwkClassInstructor}}
	\vspace{3in}
}

\author{\hmwkAuthorName}
\date{}


% Begin document
\begin{document}
\maketitle
\pagebreak
\pagebreak

% Homework problem 1
\section*{Problem 1: Definitions}
\begin{enumerate}[(a)]
  \item \textbf{Interaction plot:}
  \item \textbf{Random effect:}
  \item \textbf{Factorial design:}
  \item \textbf{Intraclass correlation:}
  \item \textbf{Split-plot design:}
\end{enumerate}

\section*{Problem 2: Marketing Problem}

A marketing research consultant evaluated the effects of fee schedule, scope of work, and
type of supervisory control on the quality of work performed under contract by independent marketing
research agencies. The quality of work performed was measured by an index taking into account several
characteristics of quality. Four agencies were chosen for each factor level combination and the quality of
their work evaluated. See \textbf{marketing.txt} files for this dataset.

\begin{enumerate}[(a)]
  \item Create interaction plots of the factors for the fee schedule and scope of the work of the 
    estimated treatment means $\bar{Y}_{ij}$. Does it appear that any interactions are present? Any main effects?
  \item Create interaction plots of the factors fee schedule and type of supervisory control of the 
    estimated treatment  means $\bar{Y}_{ijk}$. Do your plots convey the same information as those in part (a)? 
  \item Obtain the analysis of variance table. 
  \item Test for three-factor interactions: using $\alpha = 0.01$. State the alternatives, division rule, and conclusion. 
    What is the p-value of the test?
  \item Test for two-factor interactions (there are three test). For each test, use $\alpha = 0.01$ and state the alternative, decision rule, and conclusions. 
    What is the p-value of each test?
  \item Test for factor fee schedule main effects; use $\alpha= 0.05$. State the alternatives, decision rule, and conclusion. Hint: You can use the confidence interval. 
\end{enumerate}

\section*{Problem 3: Calculator}

To test the efficiency of its new programmable calculator, a computer company selected at 
random six engineers who ewere proficient in the use of both this calculator and an earlier model and 
asked them to work out two problems on tboth calculators. One of the problems was statistical in nature, 
the other was an 

\begin{enumerate}[(a)]
  \item 
\end{enumerate}


\end{document}
