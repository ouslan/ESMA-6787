\documentclass{article}

\usepackage{fancyhdr}
\usepackage{extramarks}
\usepackage{amsmath}
\usepackage{amsthm}
\usepackage{amsfonts}
\usepackage{tikz}
\usepackage[plain]{algorithm}
\usepackage{algpseudocode}
\usepackage{listings}
\usepackage{booktabs}
\usepackage{xcolor}
\usepackage[english]{babel}
\usepackage[T1]{fontenc}
\usepackage{lmodern,mathrsfs}
\usepackage{xparse}
\usepackage[inline,shortlabels]{enumitem}
\setlist{topsep=2pt,itemsep=2pt,parsep=0pt,partopsep=0pt}
\usepackage[dvipsnames]{xcolor}
\usepackage[utf8]{inputenc}
\usepackage[a4paper,top=0.5in,bottom=0.2in,left=0.5in,right=0.5in,footskip=0.3in,includefoot]{geometry}
\usepackage[most]{tcolorbox}
\tcbuselibrary{minted} % tcolorbox minted library, required to use the "minted" tcb listing engine (this library is not loaded by the option [most])
\usepackage{minted} % Allows input of raw code, such as Python code
% \usepackage[colorlinks]{hyperref}


\usetikzlibrary{automata,positioning}

\tcbset{
    pythoncodebox/.style={
        enhanced jigsaw,breakable,
        colback=gray!10,colframe=gray!20!black,
        boxrule=1pt,top=2pt,bottom=2pt,left=2pt,right=2pt,
        sharp corners,before skip=10pt,after skip=10pt,
        attach boxed title to top left,
        boxed title style={empty,
            top=0pt,bottom=0pt,left=2pt,right=2pt,
            interior code={\fill[fill=tcbcolframe] (frame.south west)
                --([yshift=-4pt]frame.north west)
                to[out=90,in=180] ([xshift=4pt]frame.north west)
                --([xshift=-8pt]frame.north east)
                to[out=0,in=180] ([xshift=16pt]frame.south east)
                --cycle;
            }
        },
        title={#1}, % Argument of pythoncodebox specifies the title
        fonttitle=\sffamily\bfseries
    },
    pythoncodebox/.default={}, % Default is No title
    %%% Starred version has no frame %%%
    pythoncodebox*/.style={
        enhanced jigsaw,breakable,
        colback=gray!10,coltitle=gray!20!black,colbacktitle=tcbcolback,
        frame hidden,
        top=2pt,bottom=2pt,left=2pt,right=2pt,
        sharp corners,before skip=10pt,after skip=10pt,
        attach boxed title to top text left={yshift=-1mm},
        boxed title style={empty,
            top=0pt,bottom=0pt,left=2pt,right=2pt,
            interior code={\fill[fill=tcbcolback] (interior.south west)
                --([yshift=-4pt]interior.north west)
                to[out=90,in=180] ([xshift=4pt]interior.north west)
                --([xshift=-8pt]interior.north east)
                to[out=0,in=180] ([xshift=16pt]interior.south east)
                --cycle;
            }
        },
        title={#1}, % Argument of pythoncodebox specifies the title
        fonttitle=\sffamily\bfseries
    },
    pythoncodebox*/.default={}, % Default is No title
}

% Custom tcolorbox for Python code (not the code itself, just the box it appears in)
\newtcolorbox{pythonbox}[1][]{pythoncodebox=#1}
\newtcolorbox{pythonbox*}[1][]{pythoncodebox*=#1} % Starred version has no frame

% Basic Document Settings
\topmargin=-0.45in
\evensidemargin=0in
\oddsidemargin=0in
\textwidth=6.5in
\textheight=9.0in
\headsep=0.25in
\linespread{1.1}

\pagestyle{fancy}
\lhead{\hmwkAuthorName}
\chead{\hmwkClass\ (\hmwkClassInstructor): \hmwkTitle}
\rhead{\firstxmark}
\lfoot{\lastxmark}
\cfoot{\thepage}
\renewcommand\headrulewidth{0.4pt}
\renewcommand\footrulewidth{0.4pt}
\setlength\parindent{0pt}

% Homework Details
\newcommand{\hmwkTitle}{Exam 2}
\newcommand{\hmwkDueDate}{December 14, 2025}
\newcommand{\hmwkClass}{ESMA 6787}
\newcommand{\hmwkClassInstructor}{Israel Almodovar}
\newcommand{\hmwkAuthorName}{\textbf{Alejandro Ouslan}}

% Title Page
\title{
	\vspace{2in}
	\textmd{\textbf{\hmwkClass:\ \hmwkTitle}}\\
	\normalsize\vspace{0.1in}\small{Due\ on\ \hmwkDueDate}\\
	\vspace{0.1in}\large{\textit{\hmwkClassInstructor}}
	\vspace{3in}
}

\author{\hmwkAuthorName}
\date{}


% Begin document
\begin{document}
\maketitle
\pagebreak
\pagebreak

% Homework problem 1
\section*{Problem 1: Definitions}
\begin{enumerate}[(a)]
	\item \textbf{Interaction plot:} An interaction plot as a visual tool to determine if
	      there interaction between 2 factors with each other. You determine there is interaction
	      when you see the line intersect with each other.
	\item \textbf{Random effect:} When we work with block designs a random effect is the unobserved
	      effects within the blocks, that is to say each group contains its own random effect $N(0,\sigma)$
	\item \textbf{Factorial design:} Factorial design is a method to combare multiple factors with each other
	      and determine its effects on the response variable. Example we al looking at the shrinkage of type of cloth
	      and water temperature, in this experiments the type of cloth and water temperature are the factors.
	\item \textbf{Intraclass correlation:} This is the correlation that groups can have with each other.
	\item \textbf{Split-plot design:} The split-plot are design method where we mesure the effects of factors. This is
	      done by first dividing the entire group in section where each will revive a specific treatment the each group is
	      divided in sub groups that recave the other factor. This results in a multiple big groups that contain a factor and inside
	      those groups there is a sub group that are randomly assigned the factors.
\end{enumerate}

\section*{Problem 2: Marketing Problem}

A marketing research consultant evaluated the effects of fee schedule, scope of work, and
type of supervisory control on the quality of work performed under contract by independent marketing
research agencies. The quality of work performed was measured by an index taking into account several
characteristics of quality. Four agencies were chosen for each factor level combination and the quality of
their work evaluated. See \textbf{marketing.txt} files for this dataset.

\begin{enumerate}[(a)]
	\item Create interaction plots of the factors for the fee schedule and scope of the work of the
	      estimated treatment means $\bar{Y}_{ij}$. Does it appear that any interactions are present? Any main effects?

	      \textbf{Answer:} There appears to no interaction between schedule. The main effect appears to be that as the fee of the
	      fee increases the quality of work decrease.

	      \begin{figure}[H]
		      \centering
		      \includegraphics[width=0.5\textwidth]{assets/prob2a.png}
		      \caption{Interaction Graph}
	      \end{figure}

	\item Create interaction plots of the factors fee schedule and type of supervisory control of the
	      estimated treatment  means $\bar{Y}_{ijk}$. Do your plots convey the same information as those in part (a)?

	      \textbf{Answer:} There appears to no interaction between supervisory. The main effect appears to be that as the fee of the
	      fee increases the quality of work decrease. This seems to be the same effect of part a

	      \begin{figure}[H]
		      \centering
		      \includegraphics[width=0.5\textwidth]{assets/prob2b.png}
		      \caption{Interaction Graph}
	      \end{figure}

	\item Obtain the analysis of variance table.

	      \begin{table}[!ht]
		      \centering
		      \caption{ANOVA for the Effect of Fee and Schedule on Work Index}
		      \begin{tabular}{lrrrrr}
			      \hline
			      \textbf{Source}    & \textbf{df} & \textbf{Sum of Squares} & \textbf{Mean Square} & \textbf{F-statistic} & \textbf{p-value}         \\
			      \hline
			      C(Fee)             & 2.0         & 10044.27                & 5022.14              & 45.09                & \(3.50 \times 10^{-11}\) \\
			      C(Schedule)        & 1.0         & 1833.98                 & 1833.98              & 16.47                & \(2.11 \times 10^{-4}\)  \\
			      C(Fee):C(Schedule) & 2.0         & 1.60                    & 0.80                 & 0.01                 & \(9.93 \times 10^{-1}\)  \\
			      Residual           & 42.0        & 4678.04                 & 111.38               & -                    & -                        \\
			      \hline
		      \end{tabular}
	      \end{table}

	\item Test for three-factor interactions: using $\alpha = 0.01$. State the alternatives, division rule, and conclusion.
	      What is the p-value of the test?
	      \begin{enumerate}
		      \item $H_0$: There is no three way interaction
		      \item $H_a$: There is three way interaction
	      \end{enumerate}
	      \begin{table}[!ht]
		      \centering
		      \caption{ANOVA for the Effect of Fee, Schedule, and Supervisory on Work Index (Part 1)}
		      \begin{tabular}{lrrrr}
			      \hline
			      \textbf{Source}                   & \textbf{df} & \textbf{Sum of Squares} & \textbf{Mean Square} \\
			      \hline
			      C(Fee)                            & 2.0         & 10044.27                & 5022.14              \\
			      C(Schedule)                       & 1.0         & 1833.98                 & 1833.98              \\
			      C(Supervisory)                    & 1.0         & 3832.40                 & 3832.40              \\
			      C(Fee):C(Schedule)                & 2.0         & 1.60                    & 0.80                 \\
			      C(Fee):C(Supervisory)             & 2.0         & 0.79                    & 0.39                 \\
			      C(Schedule):C(Supervisory)        & 1.0         & 574.78                  & 574.78               \\
			      C(Fee):C(Schedule):C(Supervisory) & 2.0         & 3.94                    & 1.97                 \\
			      Residual                          & 36.0        & 266.14                  & 7.39                 \\
			      \hline
		      \end{tabular}
	      \end{table}

	      \pagebreak

	      \begin{table}[!ht]
		      \centering
		      \caption{ANOVA for the Effect of Fee, Schedule, and Supervisory on Work Index (Part 2)}
		      \begin{tabular}{lrr}
			      \hline
			      \textbf{Source}                   & \textbf{F-statistic} & \textbf{p-value}         \\
			      \hline
			      C(Fee)                            & 679.34               & \(2.59 \times 10^{-29}\) \\
			      C(Schedule)                       & 248.08               & \(1.00 \times 10^{-17}\) \\
			      C(Supervisory)                    & 518.40               & \(5.74 \times 10^{-23}\) \\
			      C(Fee):C(Schedule)                & 0.11                 & \(8.98 \times 10^{-1}\)  \\
			      C(Fee):C(Supervisory)             & 0.05                 & \(9.48 \times 10^{-1}\)  \\
			      C(Schedule):C(Supervisory)        & 77.75                & \(1.60 \times 10^{-10}\) \\
			      C(Fee):C(Schedule):C(Supervisory) & 0.27                 & \(7.67 \times 10^{-1}\)  \\
			      Residual                          & -                    & -                        \\
			      \hline
		      \end{tabular}
	      \end{table}

	      \textbf{Answer:} Given that C(Fee):C(Schedule):C(Supervisory) as a p-value of $0.67$ we don't have enough evidence to
	      reject the null hypothesis thous we conclude there is not sufficient evidence to conclude that there is a three way interaction

	\item Test for factor fee schedule main effects; use $\alpha= 0.05$. State the alternatives, decision rule, and conclusion. Hint: You can use the confidence interval.
	      \begin{enumerate}
		      \item $H_0$: The fee has no impact on work productivity.
		      \item $H_a$: The fee has an impact on work productivity.
		      \item $H_0$: The schedule has no impact on work productivity.
		      \item $H_a$: The schedule has an impact on work productivity.
	      \end{enumerate}
\end{enumerate}
\begin{table}[!ht]
	\centering
	\caption{ANOVA for the Main Effects of Fee and Schedule on Work Index}
	\begin{tabular}{lrrrrr}
		\hline
		\textbf{Source} & \textbf{df} & \textbf{Sum of Squares} & \textbf{Mean Square} & \textbf{F-statistic} & \textbf{p-value}         \\
		\hline
		C(Fee)          & 2.0         & 10044.27                & 5022.14              & 47.22                & \(1.12 \times 10^{-11}\) \\
		C(Schedule)     & 1.0         & 1833.98                 & 1833.98              & 17.24                & \(1.49 \times 10^{-4}\)  \\
		Residual        & 44.0        & 4679.65                 & 106.36               & -                    & -                        \\
		\hline
	\end{tabular}
\end{table}

\begin{table}[!ht]
	\centering
	\caption{Confidence Intervals for the Factors in the ANOVA Model}
	\begin{tabular}{lrr}
		\hline
		\textbf{Factor}  & \textbf{Lower Bound} & \textbf{Upper Bound} \\
		\hline
		Intercept        & 110.91               & 122.91               \\
		C(Fee)[T.2]      & -8.31                & 6.39                 \\
		C(Fee)[T.3]      & -38.50               & -23.81               \\
		C(Schedule)[T.2] & -18.36               & -6.36                \\
		\hline
	\end{tabular}
\end{table}

\textbf{Answer:} Looking at the p-values both the fee and the schedule and the fee have a negative impact of the work
productivity. Looking at the confidence intervals the fee 3 seems to have the negative impact while level 2 does not
seem to have an impact given that it does not include 0.

\section*{Problem 3: Calculator}

To test the efficiency of its new programmable calculator, a computer company selected at
random six engineers who were proficient in the use of both this calculator and an earlier model and
asked them to work out two problems on both calculators. One of the problems was statistical in nature,
the other was an engineering problem. The ercer of the four calculations was randomized independently
for each engineer. The length of time (in minutes) required to solve each problem was observed. See
Table 1


\begin{table}[!ht]
	\centering
	\caption{The length of time (in minutes) required to solve each problem}
	\begin{tabular}{c | c c | c c}
		\hline
		\textbf{Programmer} & \multicolumn{2}{c|}{\textbf{Statistical Problem}} & \multicolumn{2}{c}{\textbf{Engineering Problem}}                                               \\
		\hline
		                    & \textbf{New Model}                                & \textbf{Earlier Model}                           & \textbf{New Model} & \textbf{Earlier Model} \\
		\hline
		1                   & 3.1                                               & 7.5                                              & 2.5                & 5.1                    \\
		2                   & 3.8                                               & 8.1                                              & 2.8                & 5.3                    \\
		3                   & 3.0                                               & 7.6                                              & 2.0                & 4.9                    \\
		4                   & 3.4                                               & 7.8                                              & 2.7                & 5.5                    \\
		5                   & 3.3                                               & 6.9                                              & 2.5                & 5.4                    \\
		6                   & 3.6                                               & 7.8                                              & 2.4                & 4.8                    \\
		\hline
	\end{tabular}
\end{table}



\begin{enumerate}[(a)]
	\item Create interaction plots of the estimated treatment means. Does it appear that treatment
	      interaction effects are present?
	      \begin{figure}[H]
		      \centering
		      \includegraphics[width=0.5\textwidth]{assets/prob3a.png}
		      \caption{Interaction Graph}
	      \end{figure}

	      \textbf{Answer:} There apear to be no interaction between model type and problem. There apear to be a reduction in the completion time from a stats
	      problem to a engineering problem. The overall effect is that the new model reduced the completion time.

	\item Create the skeleton ANOVA. If there are any random effects mark them.


	      \begin{table}[ht]
		      \centering
		      \caption{Skeleton ANOVA Table for Mixed Effects Model}
		      \begin{tabular}{lcccc}
			      \hline
			      \textbf{Source of Variation}                 & \textbf{df}             & \textbf{SS}         & \textbf{MS}         & \textbf{F}         \\
			      \hline
			      \textbf{Fixed Effects}                       &                         &                     &                     &                    \\
			      Model (C(Model))                             & 1                       & SS\(_{M}\)          & MS\(_{M}\)          & F\(_{M}\)          \\
			      Problem (C(Problem))                         & 1                       & SS\(_{Problem}\)    & MS\(_{P}\)          & F\(_{P}\)          \\
			      Model $\times$ Problem (C(Model):C(Problem)) & 1                       & SS\(_{M \times P}\) & MS\(_{M \times P}\) & F\(_{M \times P}\) \\
			      \hline
			      \textbf{Random Effects (R)}                  &                         &                     &                     &                    \\
			      Programmer (random)                          & \(b-1 = 5\)             & SS\(_{Prog}\)       & MS\(_{Prog}\)       & —                  \\
			      Residual/Error (within Programmer)           & \(ab - a - b + 1 = 18\) & SS\(_{R}\)          & MS\(_{R}\)          & —                  \\
			      \hline
			      Total                                        & \(N-1 = 23\)            & SS\(_{Total}\)      & —                   & —                  \\
			      \hline
		      \end{tabular}
	      \end{table}


	      \pagebreak

	\item Obtain the analysis of variance table.

	      \begin{table}[!ht]
		      \centering
		      \caption{Mixed Linear Model Fixed Effects Results}
		      \begin{tabular}{lcccccc}
			      \hline
			      \textbf{Effect}                    & \textbf{Coef.} & \textbf{Std. Err.} & \textbf{z-value} & \textbf{p-value} & \textbf{95\% CI} \\
			      \hline
			      Intercept                          & 5.167          & 0.131              & 39.402           & 0.000            & [4.910, 5.424]   \\
			      C(Model)[T.New]                    & -2.683         & 0.150              & -17.911          & 0.000            & [-2.977, -2.390] \\
			      C(Problem)[T.Stat]                 & 2.450          & 0.150              & 16.354           & 0.000            & [2.156, 2.744]   \\
			      C(Model)[T.New]:C(Problem)[T.Stat] & -1.567         & 0.212              & -7.394           & 0.000            & [-1.982, -1.151] \\
			      \hline
		      \end{tabular}
	      \end{table}


	\item Test Whether or not the two treatment factors interact. State the alternatives, decision
	      rule, and conclusion. What is the p-value of the test?
	      \begin{enumerate}
		      \item $H_0$: There is no interaction
		      \item $H_a$: There is interaction
	      \end{enumerate}
	      \textbf{Answer:} Given the previous table we see that the p value of $0.00$ and it is less than the critical value $0.05$ we reject the null hypothesis
	      and conclude that the time to solve the problem depends on the type of problem
\end{enumerate}

\section*{Problem 4: Automobile Manufacture}

An automobile manufacturer wished to study the effects of difference between drivers and
differences between cars on gasoline consumption. Four drivers were selected at random; also live cars of
the same model with manual transmission were randomly selected from the assembly line. Each driver
drove each car twice over a 40-mile test course and the miles per gallon were recorded. See Table 2

\begin{table}[!ht]
	\centering
	\caption{Miles per gallon for each driver in each car}
	\begin{tabular}{c c c c c c}
		\hline
		\textbf{Driver} & \textbf{Car 1} & \textbf{Car 2} & \textbf{Car 3} & \textbf{Car 4} & \textbf{Car 5} \\
		\hline
		1               & 25.3           & 28.9           & 24.8           & 28.4           & 27.1           \\
		1               & 25.2           & 30.0           & 25.1           & 27.9           & 26.6           \\
		\hline
		2               & 33.6           & 36.7           & 31.7           & 35.6           & 33.7           \\
		2               & 32.9           & 36.5           & 31.9           & 35.0           & 33.9           \\
		\hline
		3               & 27.7           & 30.7           & 26.9           & 29.7           & 29.2           \\
		3               & 28.5           & 30.4           & 26.3           & 30.2           & 28.9           \\
		\hline
		4               & 29.2           & 32.4           & 27.7           & 31.8           & 30.3           \\
		4               & 29.3           & 32.4           & 28.9           & 30.7           & 29.9           \\
		\hline
	\end{tabular}
\end{table}

\begin{enumerate}[(a)]
	\item Test whether or not the two factors interact. State the alternative, decision rule, and
	      conclusion. What is the p-value of the test?
	      \begin{enumerate}
		      \item $H_0$: There is no intereaciton between driver and car
		      \item $H_a$: There is interaciton between driver and car
	      \end{enumerate}

	      \begin{table}[!ht]
		      \centering
		      \caption{ANOVA Results for Two-Way Interaction (Driver and Car)}
		      \begin{tabular}{c c c c c c}
			      \hline
			      \textbf{Source}  & \textbf{df} & \textbf{Sum of Squares} & \textbf{Mean Squares} & \textbf{F} & \textbf{PR(>F)} \\
			      \hline
			      C(Driver)        & 3.0         & 280.28475               & 93.428250             & 531.597440 & 3.125304e-19    \\
			      C(Car)           & 4.0         & 94.71350                & 23.678375             & 134.727596 & 3.663600e-14    \\
			      C(Driver):C(Car) & 12.0        & 2.44650                 & 0.203875              & 1.160028   & 3.714839e-01    \\
			      Residual         & 20.0        & 3.51500                 & 0.175750              & NaN        & NaN             \\
			      \hline
		      \end{tabular}
	      \end{table}

	      \textbf{Answer:} Given that the p-value is 0.37 and it is grater than the critical value we don't have sufficient evidence
	      to reject the null hypothesis and say there is interaction present.

	\item Test separately whether or not factor A and factor B main effects are present. For each
	      test, and state the alternatives, decision rule, and conclusion. What is the p-value for each test?

	      \textbf{Driver effect}
	      \begin{enumerate}
		      \item $H_0$: The miles per gallon for each driver is the same
		      \item $H_a$: At least one is different
	      \end{enumerate}

	      \textbf{Car effect}
	      \begin{enumerate}
		      \item $H_0$: The miles per gallon for each car is the same
		      \item $H_a$: At least one is different
	      \end{enumerate}


	      \begin{table}[!ht]
		      \centering
		      \caption{ANOVA Results for Gasoline Consumption (Driver and Car)}
		      \begin{tabular}{c c c c c c}
			      \hline
			      \textbf{Source} & \textbf{df} & \textbf{Sum of Squares} & \textbf{Mean Squares} & \textbf{F} & \textbf{PR(>F)} \\
			      \hline
			      C(Driver)       & 3.0         & 280.28475               & 93.42825              & 501.502    & 5.728485e-27    \\
			      C(Car)          & 4.0         & 94.71350                & 23.67838              & 127.100    & 3.668485e-19    \\
			      Residual        & 32.0        & 5.96150                 & 0.18630               & NaN        & NaN             \\
			      \hline
		      \end{tabular}
	      \end{table}

	      \textbf{Answer:} Looking at the table both Driver and Car have p-values below the critical value of 0.05 thus we reject the null
	      hypothesis and conclude there is both a driver effect and a car effect.

	\item Obtain point estimates of and which factor appears to have the greater effect on gasoline consumption?
	      \begin{table}[!ht]
		      \centering
		      \caption{Driver Means}
		      \begin{tabular}{c|c}
			      \hline
			      \textbf{Driver} & \textbf{Mean MPG} \\
			      \hline
			      1               & 26.93             \\
			      2               & 34.15             \\
			      3               & 28.85             \\
			      4               & 30.26             \\
			      \hline
		      \end{tabular}
	      \end{table}
	      \begin{table}[!ht]
		      \centering
		      \caption{Car Means}
		      \begin{tabular}{c|c}
			      \hline
			      \textbf{Car} & \textbf{Mean MPG} \\
			      \hline
			      Car\_1       & 28.96             \\
			      Car\_2       & 32.25             \\
			      Car\_3       & 27.91             \\
			      Car\_4       & 31.16             \\
			      Car\_5       & 29.95             \\
			      \hline
		      \end{tabular}
	      \end{table}

	      \textbf{Answer:} Looking at the means driver 2 and car 2 produce the most MPG consumption. Another way you could determine this is using the
	      OLS and looking at the coaliciones it generates and seeing which has the biggest coef

	\item Use the Satterhwaite procedure to obtain an approximate 95 percent confidence interval.
	      Is your interval estimate reasonably precise? Comment

	      \begin{table}[!ht]
		      \centering
		      \begin{tabular}{c|c|c}
			      \hline
			      \textbf{Parameter} & \textbf{Lower Bound} & \textbf{Upper Bound} \\
			      \hline
			      Intercept          & 25.452               & 26.238               \\
			      C(Driver)[T.2]     & 6.827                & 7.613                \\
			      C(Driver)[T.3]     & 1.527                & 2.313                \\
			      C(Driver)[T.4]     & 2.937                & 3.723                \\
			      C(Car)[T.Car2]     & 2.848                & 3.727                \\
			      C(Car)[T.Car3]     & -1.490               & -0.610               \\
			      C(Car)[T.Car4]     & 1.760                & 2.640                \\
			      C(Car)[T.Car5]     & 0.548                & 1.427                \\
			      \hline
		      \end{tabular}
		      \caption{95\% Confidence Intervals for Model Parameters}
		      \label{tab:conf_intervals}
	      \end{table}

	      \begin{table}[!ht]
		      \centering
		      \begin{tabular}{c|c|c}
			      \hline
			      \textbf{Parameter} & \textbf{Lower Bound} & \textbf{Upper Bound} \\
			      \hline
			      Intercept          & 25.30                & 26.39                \\
			      C(Driver)[T.1]     & 6.80                 & 7.64                 \\
			      C(Driver)[T.2]     & 1.44                 & 2.40                 \\
			      C(Driver)[T.3]     & 2.86                 & 3.80                 \\
			      C(Car)[T.Car1]     & 2.73                 & 3.85                 \\
			      C(Car)[T.Car2]     & -1.58                & -0.52                \\
			      C(Car)[T.Car3]     & 1.68                 & 2.72                 \\
			      C(Car)[T.Car4]     & 0.52                 & 1.46                 \\
			      \hline
		      \end{tabular}
		      \caption{95\% Confidence Intervals with Robust Standard Errors}
		      \label{tab:conf_intervals_robust}
	      \end{table}

	      \textbf{Answer:} they are some what more precise but the not as much as i would expect or reasonably argue.

\end{enumerate}

\section*{Problem 5: Derive estimators}
Some students (Junior, Ariana, Yeily, etc.) mentioned they wanted to derived some of the equations.
Consider the following model $Y_{jik} = \mu + \alpha_i + \beta_j + \mu_{ij} + \epsilon_{ijk}; k=1,\ldots,K;i=1,2$.
Further, assume that $\mu_{ij}\sim N(0,\sigma^2_u)$ and $\epsilon_{ijk} \sim N(0,\sigma^2_e)$

\begin{enumerate}
	\item Show that,
	      $$
		      Var(Y_{ijk}) = \sigma^2_u + \sigma^2_e
	      $$
	      \textbf{Answer:}
	      \[
		      \begin{split}
			      Var(Y_{ijk}) & = Var(\mu + \alpha_i + \beta_j + \mu_{ij} + \epsilon_{ijk})                     \\
			                   & = Var(\mu) + Var(\alpha_i) + var(\beta_j) + var(\mu_{ij}) + var(\epsilon_{ijk}) \\
			                   & = 0 + 0 + 0 + \sigma^2_u + \sigma^2_e                                           \\
			                   & = \boxed{\sigma^2_u + \sigma^2_e}                                               \\
		      \end{split}
	      \]
	\item Show that, $k \neq k'$
	      $$
		      Cov(Y_{ijk}, Y_{ijk'}) = \sigma^2_u
	      $$
	      \textbf{Answer:}
	      \[
		      \begin{split}
			      Cov(Y_{ijk}, Y_{ijk'}) & = E[(Y_{ijk} - E[Y_{ijk}])(Y_{ijk'} - E[Y_{ijk'}])]                                                               \\
			                             & = E[(\mu_{ij} + \epsilon_{ijk})(\mu_{ij} + \epsilon_{ijk})]                                                       \\
			                             & = E[(\mu_{ij}\mu_{ij})] + E[(\mu_{ij}\sigma_{ijk})] + E[(\mu_{ij}\sigma_{ijk'})] + E[(\sigma_{ijk}\sigma_{ijk'})] \\
			                             & = cov(\mu_{ij}) + 0 + 0 + 0                                                                                       \\
			                             & = \boxed{\sigma^2_i}
		      \end{split}
	      \]
	\item Show that, $k \neq k'$
	      $$
		      Corr(Y_{ijk}, Y_{ijk'}) = \frac{\sigma^2_u}{\sigma^2_u + \sigma^2_e}
	      $$
\end{enumerate}

\end{document}
